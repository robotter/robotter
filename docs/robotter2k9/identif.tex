%%-----------------------------------------------------------------------------

\mychapter{Identification du capteur ADNS}

\section{Th�orie}

Le capteur de position constitu� de 3 capteurs optiques ADNS6010 n'est pas
parfaitement ajust� dans les axes de la base roulante et il est, par cons�quent,
difficile de calculer la matrice $A$ vue pr�c�dement a partir des mesures du
capteur.

\begin{equation}
\left(\begin{array}{c} 
v_x\\v_y\\\omega_z\end{array}\right)
=A
\left(\begin{array}{c}
v_{x1}\\v_{y1}\\v_{x2}\\v_{y2}\\v_{x3}\\v_{y3}\end{array}\right)
\end{equation}


La m�thode est donc d'effectuer une s�rie de mesure de couples $(U,V)$ o� 
$U = \left(\begin{array}{c}v_x\\v_y\\\omega_z\end{array}\right)$ et
$V = \left(\begin{array}{c}v_{x1}\\v_{y1}\\v_{x2}\\v_{y2}\\v_{x3}\\
v_{y3}\end{array}\right)$.\\

Par exemple effectuer depuis le point
$\left(\begin{array}{c}0\\0\\0\end{array}\right)$ 
une translation de $400 mm$ suivant l'axe $\vec{x}$ donne 
le couple $(U_0,V_0)$ suivant :

\begin{equation}
\left(
\left(\begin{array}{c}400\\0\\0\end{array}\right)
,
\left(\begin{array}{c}30110\\-17327\\-30676\\-17198\\972\\35188
\end{array}\right)
\right)
\end{equation}

Il reste � effectuer un grand nombre de mesure sur des
coordonn�es vari�es et d'ensuite identifier la matrice $A$ au moyen d'un
algorithme d'identification.

\section{Code}

L'identification est effectu�e au moyen de la m�thode ARMA sous SciLab
gr�ce au script\\\verb$unioc_asserv/scilab/adns_calibration.sci$.

